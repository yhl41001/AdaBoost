\section{Introduzione}

Il riconoscimento facciale è stato oggetto di ricerca dai primi anni novanta come parte del ramo del \emph{pattern recognition} e della \emph{object detection}. Il riconoscimento di oggetti e di facce presentano circa gli stessi problemi, ma in particolare, nel secondo caso, i problemi più determinanti sono dati dalla differenza di illuminazione, la scala e la rotazione del volto. 

A questi problemi classici del riconoscimento di oggetti se ne aggiungono altri come la gran varietà di espressioni del volto umano o alla presenza di elementi che ne potrebbero compromettere il riconoscimento, come ad esempio una particolare capigliatura o semplicemente un paio di occhiali.

L'obiettivo di questo elaborato è implementare il metodo di riconoscimento facciale proposto da Viola e Jones \cite{Viola:2004:RRF:966432.966458} e valutarne le criticità e performance.